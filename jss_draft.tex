% Options for packages loaded elsewhere
\PassOptionsToPackage{unicode}{hyperref}
\PassOptionsToPackage{hyphens}{url}
\PassOptionsToPackage{dvipsnames,svgnames,x11names}{xcolor}
%
\documentclass[
  article]{jss}

\usepackage{amsmath,amssymb}
\usepackage{iftex}
\ifPDFTeX
  \usepackage[T1]{fontenc}
  \usepackage[utf8]{inputenc}
  \usepackage{textcomp} % provide euro and other symbols
\else % if luatex or xetex
  \usepackage{unicode-math}
  \defaultfontfeatures{Scale=MatchLowercase}
  \defaultfontfeatures[\rmfamily]{Ligatures=TeX,Scale=1}
\fi
\usepackage{lmodern}
\ifPDFTeX\else  
    % xetex/luatex font selection
\fi
% Use upquote if available, for straight quotes in verbatim environments
\IfFileExists{upquote.sty}{\usepackage{upquote}}{}
\IfFileExists{microtype.sty}{% use microtype if available
  \usepackage[]{microtype}
  \UseMicrotypeSet[protrusion]{basicmath} % disable protrusion for tt fonts
}{}
\makeatletter
\@ifundefined{KOMAClassName}{% if non-KOMA class
  \IfFileExists{parskip.sty}{%
    \usepackage{parskip}
  }{% else
    \setlength{\parindent}{0pt}
    \setlength{\parskip}{6pt plus 2pt minus 1pt}}
}{% if KOMA class
  \KOMAoptions{parskip=half}}
\makeatother
\usepackage{xcolor}
\setlength{\emergencystretch}{3em} % prevent overfull lines
\setcounter{secnumdepth}{-\maxdimen} % remove section numbering
% Make \paragraph and \subparagraph free-standing
\makeatletter
\ifx\paragraph\undefined\else
  \let\oldparagraph\paragraph
  \renewcommand{\paragraph}{
    \@ifstar
      \xxxParagraphStar
      \xxxParagraphNoStar
  }
  \newcommand{\xxxParagraphStar}[1]{\oldparagraph*{#1}\mbox{}}
  \newcommand{\xxxParagraphNoStar}[1]{\oldparagraph{#1}\mbox{}}
\fi
\ifx\subparagraph\undefined\else
  \let\oldsubparagraph\subparagraph
  \renewcommand{\subparagraph}{
    \@ifstar
      \xxxSubParagraphStar
      \xxxSubParagraphNoStar
  }
  \newcommand{\xxxSubParagraphStar}[1]{\oldsubparagraph*{#1}\mbox{}}
  \newcommand{\xxxSubParagraphNoStar}[1]{\oldsubparagraph{#1}\mbox{}}
\fi
\makeatother


\providecommand{\tightlist}{%
  \setlength{\itemsep}{0pt}\setlength{\parskip}{0pt}}\usepackage{longtable,booktabs,array}
\usepackage{calc} % for calculating minipage widths
% Correct order of tables after \paragraph or \subparagraph
\usepackage{etoolbox}
\makeatletter
\patchcmd\longtable{\par}{\if@noskipsec\mbox{}\fi\par}{}{}
\makeatother
% Allow footnotes in longtable head/foot
\IfFileExists{footnotehyper.sty}{\usepackage{footnotehyper}}{\usepackage{footnote}}
\makesavenoteenv{longtable}
\usepackage{graphicx}
\makeatletter
\def\maxwidth{\ifdim\Gin@nat@width>\linewidth\linewidth\else\Gin@nat@width\fi}
\def\maxheight{\ifdim\Gin@nat@height>\textheight\textheight\else\Gin@nat@height\fi}
\makeatother
% Scale images if necessary, so that they will not overflow the page
% margins by default, and it is still possible to overwrite the defaults
% using explicit options in \includegraphics[width, height, ...]{}
\setkeys{Gin}{width=\maxwidth,height=\maxheight,keepaspectratio}
% Set default figure placement to htbp
\makeatletter
\def\fps@figure{htbp}
\makeatother

\usepackage{orcidlink,thumbpdf,lmodern}

\newcommand{\class}[1]{`\code{#1}'}
\newcommand{\fct}[1]{\code{#1()}}
\makeatletter
\@ifpackageloaded{caption}{}{\usepackage{caption}}
\AtBeginDocument{%
\ifdefined\contentsname
  \renewcommand*\contentsname{Table of contents}
\else
  \newcommand\contentsname{Table of contents}
\fi
\ifdefined\listfigurename
  \renewcommand*\listfigurename{List of Figures}
\else
  \newcommand\listfigurename{List of Figures}
\fi
\ifdefined\listtablename
  \renewcommand*\listtablename{List of Tables}
\else
  \newcommand\listtablename{List of Tables}
\fi
\ifdefined\figurename
  \renewcommand*\figurename{Figure}
\else
  \newcommand\figurename{Figure}
\fi
\ifdefined\tablename
  \renewcommand*\tablename{Table}
\else
  \newcommand\tablename{Table}
\fi
}
\@ifpackageloaded{float}{}{\usepackage{float}}
\floatstyle{ruled}
\@ifundefined{c@chapter}{\newfloat{codelisting}{h}{lop}}{\newfloat{codelisting}{h}{lop}[chapter]}
\floatname{codelisting}{Listing}
\newcommand*\listoflistings{\listof{codelisting}{List of Listings}}
\makeatother
\makeatletter
\makeatother
\makeatletter
\@ifpackageloaded{caption}{}{\usepackage{caption}}
\@ifpackageloaded{subcaption}{}{\usepackage{subcaption}}
\makeatother
\makeatletter
\@ifpackageloaded{tcolorbox}{}{\usepackage[skins,breakable]{tcolorbox}}
\makeatother
\makeatletter
\@ifundefined{shadecolor}{\definecolor{shadecolor}{rgb}{.97, .97, .97}}{}
\makeatother
\makeatletter
\makeatother
\makeatletter
\ifdefined\Shaded\renewenvironment{Shaded}{\begin{tcolorbox}[borderline west={3pt}{0pt}{shadecolor}, enhanced, boxrule=0pt, interior hidden, frame hidden, sharp corners, breakable]}{\end{tcolorbox}}\fi
\makeatother

\ifLuaTeX
  \usepackage{selnolig}  % disable illegal ligatures
\fi
\usepackage{bookmark}

\IfFileExists{xurl.sty}{\usepackage{xurl}}{} % add URL line breaks if available
\urlstyle{same} % disable monospaced font for URLs
\hypersetup{
  pdftitle={tidyclust: Implementation of Frequent Itemset Mining {[}WIP{]}},
  pdfauthor={Andrew Kerr},
  pdfkeywords={JSS, style guide, comma-separated, not capitalized, R},
  colorlinks=true,
  linkcolor={blue},
  filecolor={Maroon},
  citecolor={Blue},
  urlcolor={Blue},
  pdfcreator={LaTeX via pandoc}}


%% -- Article metainformation (author, title, ...) -----------------------------

%% Author information
\author{Andrew Kerr~\orcidlink{0000-0003-0918-3766}\\California
Polytechnic State University}
\Plainauthor{Andrew Kerr} %% comma-separated

\title{tidyclust: Implementation of Frequent Itemset Mining {[}WIP{]}}
\Plaintitle{tidyclust: Implementation of Frequent Itemset Mining
{[}WIP{]}} %% without formatting

%% an abstract and keywords
\Abstract{WIP}

%% at least one keyword must be supplied
\Keywords{JSS, style guide, comma-separated, not
capitalized, \proglang{R}}
\Plainkeywords{JSS, style guide, comma-separated, not capitalized, R}

%% publication information
%% NOTE: Typically, this can be left commented and will be filled out by the technical editor
%% \Volume{50}
%% \Issue{9}
%% \Month{June}
%% \Year{2012}
%% \Submitdate{2012-06-04}
%% \Acceptdate{2012-06-04}
%% \setcounter{page}{1}
%% \Pages{1--xx}

%% The address of (at least) one author should be given
%% in the following format:
\Address{
Andrew Kerr\\
Department of Statistics\\
959 Higuera St.\\
San Luis Obispo United States of America\\
E-mail: \email{adkerr2001@gmail.com}\\
URL: \url{https://github.com/Wander03/tidyclust}\\
\\~

}

\begin{document}
\maketitle


\section{Introduction}\label{sec-intro}

Since the early 1990s, frequent itemset mining (FIM) has been an
important area of research in data mining. The work of Agrawal,
Imieliński, and Swami \citeyearpar{agrawal_mining_1993} introduced the
Apriori algorithm, which established the foundation for identifying
co-occurring items in transactional datasets. Shortly after, algorithms
such as FP-growth's FP-tree structure \citep{han_mining_2000} and
Elcat's vertical tid-list approach \citep{zaki_scalable_2000}
significantly improved the performance and scalability of FIM
techniques. These techniques have since been applied to a wide range of
applications, including market basket analysis, web personalization
\citep{mobasher_effective_2001}, and associative classification
\citep{liu_integrating_1998}.

Although FIM has become more efficient, practical implementations remain
confined to specialized software packages. The \proglang{R}
\citep{r_core_team_r_2024} programming environment offers itemset mining
through \pkg{arules} \citep{hahsler_arules_2011}, while
\proglang{Python} users rely on \pkg{mlxtend}
\citep{raschka_mlxtend_2018} for similar functionality. However, these
implementations share the same limitation: they operate as stand alone
itemset mining applications rather than as integrated components in a
modern data science workflow. The absence of native integration with
frameworks like \pkg{tidymodels} \citep{kuhn_tidymodels_2020} or
\pkg{scikit-learn} \citep{buitinck_api_2013} leaves a gap, forcing
analysts to create their own custom solutions.

Another limitation lies in the interpretation of FIM results. Frequent
itemsets are commonly treated as either final outputs, or inputs for
generating association rules. However, research by Cheng et al.
\citeyearpar{cheng_discriminative_2007} demonstrated that itemsets can
also be used in classification tasks, while other studies have explored
how itemsets can be used in clustering tasks
\citep{mobasher_effective_2001, wickramaratna_predicting_2009}. Although
these approaches lack a standardized method, the use of FIM in these
methods indicates a potential for integration with clustering workflows.

This thesis seeks to address these limitations by introducing FIM to
\pkg{tidyclust} \citep{hvitfeldt_tidyclust_2022}, a package designed for
unsupervised learning algorithms within the \pkg{tidymodel} framework.
By adapting the Apriori and Eclat algorithms to \pkg{tidyclust}, this
work allows itemset mining to be used within unsupervised workflows. It
also creates a standard methodology for predicting missing-items, and
lays a foundation for future additions of mining and clustering
techniques within \pkg{tidyclust}.

\section{Models and software}\label{sec-models}

\subsection{Frequent Itemset Mining}\label{frequent-itemset-mining}

Frequent itemset mining (FIM) methods were developed to identify the
elements of a transactional dataset that often occur together. The
original use is in market basket research \citep{agrawal_mining_1993},
where FIM techniques help discover items commonly bought together. For
example, a pattern could be that milk and eggs are frequently bought
together in the same transaction. This type of information can be used
by the shop owner to place these items apart from each other, causing
customers to walk through more of the store. Although this task may
appear simple, the number of itemsets rapidly grows in wider datasets,
resulting in the development of more efficient algorithms. One such
development is the principle that if an itemset is frequent, then so are
all of its subsets; a property known as the Apriori principle or
downward closure property. This principle significantly reduces the
search space by systematically pruning candidate itemsets that do not
meet a minimum support threshold. The two methods implemented in
\pkg{tidyclust}, Apriori \citep{agrawal_mining_1993} and Eclat
\citep{zaki_scalable_2000}, both operate on this principle.

\subsection{tidyclust}\label{tidyclust}

The \pkg{tidyclust} package \citep{hvitfeldt_tidyclust_2022} extends the
\pkg{tidymodels} \citep{kuhn_tidymodels_2020} framework to unsupervised
tasks. These packages were built following the principles of the
\pkg{tidyverse} \citep{wickham_welcome_2019}, with the goal of
establishing a consistent and reproducible workflow. Since the design of
\pkg{tidyclust} is closely modeled off of \pkg{parsnip}
\citep{kuhn_parsnip_2024}, users are able to specify an unsupervised
learning model, then fit and predict on the model using a standardized
syntax.

\section{FIM in tidyclust}\label{sec-implementation}

\subsection{Model Specification}\label{model-specification}

The model specification for frequent itemset mining follows the same
structure as other models in \pkg{tidyclust}:

\begin{verbatim}
R>  fi_spec <- freq_itemsets(
+   min_support = 0.05, 
+   mining_method = "eclat"
+   ) %>%
+   set_engine("arules")
\end{verbatim}

The only implemented engine is \pkg{arules}. There are two arguments:
\texttt{min\_support} is the user specified minimum support value for an
itemset, taking a value between 0 and 1, and \texttt{mining\_method} is
an optional argument specifying the algorithm to use (\texttt{apriori}
or \texttt{eclat}), the default being \texttt{eclat}. Eclat was selected
as the default algorithm since it is faster in most practical cases
\citep{singla_comprehensive_2023}. A default value for
\texttt{min\_support} is not supplied since this value greatly varies
depending on the characteristics of the data.

Recipes and workflows are not impacted by any additions that I have made
to \pkg{tidyclust}. These tools are the same and work with
freq\_itemsets{[}FORMAT{]} without error.

\subsection{Cluster Assignment
Methodology}\label{cluster-assignment-methodology}

While fitting a freq\_itemsets{[}FORMAT{]} model is the same, multiple
decisions had to be made to create the desired output:

\begin{verbatim}
R> fi_fit <- fi_spec %>%
+  fit(~ .,
+    data = groceries
+  )
\end{verbatim}

A strength of the \pkg{tidyclust} framework is that the output of each
function is standardized, formatted the same way. The main product of a
fitted model are the cluster assignments, which are stored in a
\texttt{tibble} with the sole column \texttt{.cluster}.

\begin{verbatim}
R> fi_fit %>% 
+  extract_cluster_assignment()
\end{verbatim}

However, unlike other unsupervised learning algorithms (e.g., K-means)
that explicitly partition observations into distinct groups, FIM
produces a set of co-occuring items without inherent cluster labels.
Additionally, in FIM the items (columns) are of interest, while in other
unsupervised learning algorithms the observations (rows) are of
interest. The following implementation introduces a clustering
assignment strategy that assigns items to clusters based on itemset
membership and support values:

For each items in the dataset:

\begin{enumerate}
\def\labelenumi{\arabic{enumi}.}
\item
  Relevant Itemset Identification: Finds all itemsets containing the
  item.
\item
  Best Itemset Selection: Prioritizes itemsets by:

  \begin{itemize}
  \item
    Size: Larger itemsets over smaller ones (encouraging broader
    clusters).
  \item
    Support: Higher-support itemsets break ties (favoring more frequent
    patterns).
  \end{itemize}
\item
  Cluster Propagation: Assigns all items in the selected itemsets to the
  same cluster.
\end{enumerate}

If an item has already been assigned a Cluster, then the best itemset is
compared to the best itemset used for this Cluster assignment. The same
itemset prioritization is applied, and if the new best itemset is
superior then all relevant items are assigned to a new Cluster.

Items that appear in no frequent itemsets are labeled as outliers
(Cluster\_0\_X) while items within frequent itemsets are assigned
sequential cluster IDs (Cluster\_1, Cluster\_2, etc.).

Cheng et al. \citeyearpar{cheng_discriminative_2007} analyzes the
relationship between pattern frequency and discriminative power,
demonstrating that frequent patterns with higher support and larger
sizes exhibit greater predictive utility in classification tasks. These
findings are used as the reasoning for prioritizing larger itemsets with
higher supports.

\subsection{Prediction Methods}\label{prediction-methods}

Similar to the fit{[}FORMAT{]} method, while the syntax for predicting
using a fitted freq\_itemsets{[}FORMAT{]} model is the same, the input
and methodology of the predict function differs from other
\pkg{tidyclust} models:

\begin{verbatim}
R> fi_pred <- fi_fit %>%
+  predict(new_data = new_groceries)
\end{verbatim}

When deciding what it means to predict using frequent itemsets, my goal
was to create a method that has practical use in a real world setting.
This lead to implementing a recommendation system, where the new data
will have partial information about a transaction and predicts which
other items are likely to be in the transaction. For example, suppose a
customer in a grocery store items \texttt{A} and \texttt{B} in their
cart. This system would predict which other items they are likely to add
based on historical purchasing patterns where \texttt{A} and \texttt{B}
appeared together.

The implementation of the recommendation system was inspired by
traditional association rule mining such as CARs
\citep{liu_integrating_1998} and usage-based recommendations
\citep{mobasher_effective_2001}. The method predicts item-level
probabilities using the confidence from frequent itemsets, retains
interpretability from a rule-based reasoning approach, and handles
sparse data through a global support fallback. Specifically, for each
transaction in the new data:

\begin{enumerate}
\def\labelenumi{\arabic{enumi}.}
\tightlist
\item
  Missing Item Identification: Finds all missing items in the
  transaction.
\end{enumerate}

For each missing item:

\begin{enumerate}
\def\labelenumi{\arabic{enumi}.}
\item
  Relevant Itemset Identification: Finds all itemsets containing the
  item.
\item
  Context Filtering: Retain itemsets that include at least one observed
  item from the transaction.
\item
  Confidence Aggregation: For each retained itemset, compute the
  confidence of the association where
\end{enumerate}

\[
\text{confidence} = \frac{\text{support(itemset)}}{\text{support(observed items in itemset)}}
\]

\begin{enumerate}
\def\labelenumi{\arabic{enumi}.}
\setcounter{enumi}{3}
\item
  Prediction Generation: Average the confidences across all retained
  itemsets.
\item
  Fallback Handling: If there are no relevant or retained itemsets, use
  the items global support (frequency in training data) as the
  prediction. If the item does not appear in any frequent itemsets, then
  this global support will be 0.
\end{enumerate}

Aside from the \texttt{new\_data} argument, predict{[}FORMAT{]} has the
optional \texttt{type} argument which takes either
\texttt{\textquotesingle{}raw\textquotesingle{}} or
\texttt{\textquotesingle{}cluster\textquotesingle{}}. The
\texttt{\textquotesingle{}raw\textquotesingle{}} option returns the raw
prediction probabilities, while the default option
\texttt{\textquotesingle{}cluster\textquotesingle{}} applies a threshold
(0.5) to convert probabilities to binary predictions. Both options
return the predictions as a \texttt{tibble} with the sole column
\texttt{.pred\_cluster}.

While this \texttt{tibble} format is standardized across \pkg{tidyclust}
predict outputs, it does not provide the information necessary for FIM
predict output since FIM prediction on the items (columns) and not the
observations (rows). To keep this standardized formatting while
providing the necessary information, each row of the output
\texttt{tibble} represents a transaction from the new data and holds a
dataframe rather than a string or integer. This dataframe has three
columns: \texttt{item}, \texttt{.obs\_item}, and \texttt{.pred\_item}.
The \texttt{item} column represents each item (column) from the
transaction, and \texttt{.obs\_item} is \texttt{1} or \texttt{0} for any
observed items and \texttt{NA} for any missing items. The
\texttt{.pred\_item} column is the opposite of the \texttt{.obs\_item}
column, where any observed items are \texttt{NA} and any missing items
are either the raw or binary predicted values. This structure allows the
user to easily identify what values are observed or predicted in each
transaction while still containing all the information necessary to
recreate the original transaction dataset. To assist the user with
reconstruction, the function extract\_predictions{[}FORMAT{]} takes the
\texttt{tibble} output and reformats it to a single data frame.

\begin{verbatim}
R> fi_pred %>%
+  extract_predictions
\end{verbatim}

\subsection{Metrics}\label{metrics}

While the quality of frequent itemsets is traditionally assessed by
support values, support alone does not guarantee predictive performance.
Since the predict methodology follows that of a recommendation system,
the results should be evaluated using the same metrics. Common metrics
such as root mean squared error (RMSE), accuracy, precision, and recall
are implemented in the \pkg{yardstick} package
\citep{kuhn_yardstick_2024}. The inputs are a list of predicted values
and a list of the true values, and the function
augment\_itemset\_predict{[}FORMAT{]} in \pkg{tidyclust} formats the
predict output as such.

\begin{verbatim}
R> fi_fit %>%
+  predict(new_data = new_groceries, type = 'raw') %>%
+  augment_itemset_predict(pred_output, truth) %>%
+  rmse(truth, preds)
R> 
R> fi_fit %>%
+  predict(new_data = new_groceries, type = 'cluster') %>%
+  augment_itemset_predict(pred_output, truth) %>%
+  mutate(
+    truth = factor(truth, levels = c(0, 1)), 
+    preds = factor(preds, levels = c(0, 1))
+    ) %>%
+  precision(truth, preds)
\end{verbatim}

When using RMSE, the \texttt{raw} predict output should be used, while
accuracy, precision, and recall will use the \texttt{cluster} output or
user thresholded \texttt{raw} output. Caution should be used when
looking at accuracy, precision, and recall since they can be misleading
for imbalanced datasets (where items are rarely purchased). In such
cases, I recommend using F1-Score as a balance between precision and
recall and looking at precision-recall curves
\citep{saito_precision-recall_2015} to determine the best threshold
value.

\begin{verbatim}
R> fi_fit %>%
+  predict(new_data = new_groceries, type = 'raw') %>%
+  augment_itemset_predict(pred_output, truth) %>%
+  mutate(truth = factor(truth, levels = c(0, 1))) %>%
+  pr_curve(truth, preds) %>%
+  autoplot()
\end{verbatim}

\subsection{Hyperparameter Tuning}\label{hyperparameter-tuning}

The sole parameter capable of being tuned in a FIM model is
\texttt{min\_support}. As noted in the Model Specification section,
there is no default \texttt{min\_support} value since this value varies
depending on the characteristics of the dataset. Therefore, selecting
the correct value is imperative for finding useful frequent itemsets.
The default grid for \texttt{min\_support} is from 0.1 to 0.5. The lower
bound of 0.1 was chosen to avoid reporting too many frequent itemsets,
even for smaller datasets, while the upper bound of 0.5 was selected
since it ensures that at each frequent itemset has a support of at least
50\%. If this range does not work, then the user can explore their
dataset and create their own custom range of values.

\begin{verbatim}
R> min_support_grid <- grid_regular(
+  min_support(), 
+  levels = 10
+  )
\end{verbatim}

Usually the above object is paired with tune\_cluster{[}FORMAT{]},
however cross-validation is not currently implemented for FIM and will
be further discussed in Limitations and Future Directions.

\section{Limitations and Future Directions}\label{sec-limits}

While this implementation successfully integrates frequent itemset
mining into the \pkg{tidyclust} framework, there are several
limitations. This section discusses these constraints and proposes
potential enhancements to improve the methodology.

\subsection{Cluster Assignment
Methodology}\label{cluster-assignment-methodology-1}

The current clustering method prioritizes itemsets based on size and
support, assigning items to clusters by identifying the largest and most
frequent patterns. However, relying on size and support alone may not
always align with domain-specific objectives. For example, smaller but
more distinct itemsets may be more informative in applications like
fraud detection or medical diagnosis. Therefore, alternative metrics for
selecting the ``best'' itemset could improve cluster quality in these
situation.

\subsection{Prediction Methodology}\label{prediction-methodology}

The prediction implementation currently averages confidence scores
across all matching itemsets to determine. Additional methods may
improve predictive performance, such as using the maximum confidence
\citep{mobasher_effective_2001} or a weighted average by itemset size or
support. These methods could be added as an optional parameter for the
user in the predict function:

\begin{verbatim}
predict(
  fi_fit,
  new_data = new_groceries,
  confidence_agg = "max"
)
\end{verbatim}

Furthermore, these methods do not incorporate the cluster assignments
created when fitting the model. By weighting confidences higher for
itemsets where multiple observed items share cluster membership with the
predicted item, the predictive performance could be improved.

\subsection{Hyperparameter Tuning}\label{hyperparameter-tuning-1}

The current implementation requires manual specification of
\texttt{min\_support} ranges for tuning. This presents challenges for
users unfamiliar with their datasets characteristics or new to frequent
itemset mining. The \pkg{tune} package used in \pkg{tidyclust} relies on
the \pkg{dials} package \citep{kuhn_dials_2024}, which offers finalize
functions. These functions take the dataset as a parameter and modify
the unknown parts of ranges based on the dataset. Building on Dahbi et
al.'s \citeyearpar{dahbi_finding_2021} statistically-grounded approach,
I propose a two-step approach:

\begin{enumerate}
\def\labelenumi{\arabic{enumi}.}
\tightlist
\item
  Calculate the mean support of all 1-itemsets:
\end{enumerate}

\[
\mu = \frac{1}{n}\sum^n_{i=1}{\text{support}(\{\text{item}_i\})}
\]

The median or mode could also be used here.

\begin{enumerate}
\def\labelenumi{\arabic{enumi}.}
\setcounter{enumi}{1}
\tightlist
\item
  Create a confidence-interval-like range around \(\mu\) using the
  standard deviation \(\sigma\):
\end{enumerate}

\[
[\mu - \frac{\sigma}{2}, \mu + \sigma]
\]

clipping the bounds at \([0, 1]\) to ensure valid support values. An
asymmetric range was selected since transactional datasets tend to have
a right skewed support distribution.

\subsection{Evaluation Metrics}\label{evaluation-metrics}

While the current implementation works with standard metrics through
augment\_itemset\_predict{[}FORMAT{]}, specialized metrics could provide
better insight into prediction performance. The functions output
preserves both transaction-level and item-level identifiers, allowing
for two different types of analysis:

\begin{enumerate}
\def\labelenumi{\arabic{enumi}.}
\item
  Item-level metrics: measures how accurately specific items are
  predicted across all transactions.
\item
  Transaction-level metrics: evaluates complete basket prediction
  quality.
\end{enumerate}

These metrics could be used to detect consistently poor predictions for
niche items and pinpoint whether errors are from specific items or
transaction patterns.

\subsection{Cross-Validation}\label{cross-validation}

The primary objective of cross-validation in frequent itemset mining is
to determine the optimal \texttt{min\_support} threshold that maximizes
the model's predictive performance while maintaining generalizability.
However, traditional k-fold CV randomly splits transactions, which would
potentially break item co-occurrence patterns essential for FIM.
Standard k-fold CV randomly partitions the dataset into k subsets
(folds), using k-1 folds for training and the remaining fold for
validation. This procedure assumes independence between training and
validation data, but in FIM this breaks down for two reasons:

\begin{enumerate}
\def\labelenumi{\arabic{enumi}.}
\item
  Pattern Fragmentation: Randomly splitting transactions may divide
  itemset occurrences across multiple folds, artificially reducing their
  observed support leading to underestimation of their importance..
\item
  Dependence Violation: Since frequent itemsets are derived from the
  entire dataset, any validation set inherently contains information
  from the training folds through shared itemsets. This violates the
  independence assumption fundamental to CV's statistical validity.
\end{enumerate}

To address these issues, I propose item-stratified CV. This approach
constructs folds that preserve the overall frequency distribution of
items, ensuring that each fold is representative in terms of both item
frequencies and co-occurrence patterns. By stratifying on items rather
than transactions, we maintain the structural integrity of the itemsets
and prevent support distortion across folds. For datasets with temporal
components, such as purchases over time, using earlier time periods for
training and later periods for validation would reflect real-world
deployment. In both approaches, prediction is evaluated by masking a
random subset (e.g., 20\%--30\%) of items within each transaction in the
validation set. The model then predicts the masked items based on the
observed partial basket.

\section*{Computational details}\label{computational-details}
\addcontentsline{toc}{section}{Computational details}

The results in this paper were obtained using
\proglang{R}\textasciitilde4.4.1 with the
\pkg{tidyclust}\textasciitilde0.2.3.9000 package. \proglang{R} itself
and all packages used are available from the Comprehensive \proglang{R}
Archive Network (CRAN) at {[}https://CRAN.R-project.org/{]}.

\section*{Acknowledgments}\label{acknowledgments}
\addcontentsline{toc}{section}{Acknowledgments}

Dr.~Bodwin

\section*{References}\label{references}
\addcontentsline{toc}{section}{References}

\renewcommand{\bibsection}{}
\bibliography{freq_itemsets_citations.bib}





\end{document}
