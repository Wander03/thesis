\def\title{Frequent Itemset Mining with tidyclust in R}
\def\author{Andrew Davis Kerr}
\def\degreemonth{June}
\def\degreeyear{2025}
\def\degree{Master of Science in Statistics}
\def\numberofmembers{3}
   \def\chair{Kelly Bodwin, Ph.D. \linebreak Associate Professor of Statistics and Data Science}
   \def\othermemberA{Hunter Glanz, Ph.D. \linebreak Professor of Statistics and Data Science}
   \def\othermemberB{Alexander Dekhtyar, Ph.D. \linebreak Professor of Computer Science and Software Engineering}
   \def\othermemberC{Outside Advisor \linebreak CTO, External Organization Name}
   \def\othermemberD{Awesome Lecturer, M.S. \linebreak Lecturer of Some Department}
   \def\othermemberE{Faculty Member, VI, Ph.D. \linebreak Professor of Some Department}
   \def\othermemberF{Faculty Member, VII, Ph.D. \linebreak Professor of Some Department}
   \def\othermemberG{Faculty Member, VIII, Ph.D. \linebreak Professor of Some Department}
   \def\othermemberH{Faculty Member, IX, Ph.D. \linebreak Professor of Some Department}
\def\keywords{Clustering, Data Mining, Apriori, ECLAT, Frequent Itemset, tidyclust, R.}

\def\abstract{%
Unsupervised learning is closely associated with clustering, however other methods
fall under this umbrella such as data mining. In R, the tidyclust package provides a 
unified interface for clustering models, yet lacks support for data mining. This 
thesis addresses this gap by introducing the Apriori and ECLAT algorithms into tidyclust, 
with a focus on frequent itemset mining. Unlike traditional clustering models, frequent 
itemsets produce groupings of column variables, rather than cluster labels or partitions 
of observations. To address this, a novel clustering approach is proposed: items (columns) 
are grouped based on their "dominant" frequent itemset. A key contribution is a new
prediction method, modeled as a recommender system, to predict missing items. This 
implementation extends tidyclust to support column-based clustering, with applications 
in market basket analysis and recommender systems.
}

\def\tocpagelegnth{2}